%-----------------ページ余白の設定---------------------
% 左右余白は15mm固定
\usepackage[top=20truemm, bottom=20truemm, left=15truemm, right=15truemm]{geometry}

% 段間(2段の間隔)は10mm固定
\setlength{\columnsep}{10truemm}

% 本文9ptのまま「1段あたり全角26文字」相当にするため、字間(\kanjiskip)を固定で調整
% ※ 段幅は余白/段間で固定されるため、ここでは字間で調整する
\newcount\GYColSP
\newcount\GYKanjiskipSP
\newcount\GYBaseCharsSP
\newcount\GYBaseCharsPlusOneSP
\newcount\GYUpperSP
\newcount\GYLowerSP
\newcount\GYI
\newcommand{\GraduateSetKanjiskipForTwentySix}{%
  \GYColSP=\number\dimexpr\columnwidth\relax
  % 実際の和文組版幅(JFM由来の自然幅も含む)から、
  % 26文字は入る & 27文字は入らない を同時に満たす字間を算出
  \begingroup
    \kanjiskip=0pt\relax
    \setbox0=\hbox{\GYI=0\loop\ifnum\GYI<26 あ\advance\GYI by1\repeat}%
    \global\GYBaseCharsSP=\number\dimexpr\wd0\relax
    \setbox0=\hbox{\GYI=0\loop\ifnum\GYI<27 あ\advance\GYI by1\repeat}%
    \global\GYBaseCharsPlusOneSP=\number\dimexpr\wd0\relax
  \endgroup
  % Upper bound: 26 chars fit => k <= (col - base26)/25
  \GYUpperSP=\numexpr(\GYColSP-\GYBaseCharsSP)/25\relax
  \ifnum\GYUpperSP>0 \advance\GYUpperSP by -1\fi
  % Lower bound: 27 chars overflow => k > (col - base27)/26
  \GYLowerSP=\numexpr(\GYColSP-\GYBaseCharsPlusOneSP)/26\relax
  \advance\GYLowerSP by 1
  \ifnum\GYUpperSP<0 \GYUpperSP=0\fi
  \ifnum\GYLowerSP<0 \GYLowerSP=0\fi
  \GYKanjiskipSP=\GYUpperSP
  \ifnum\GYKanjiskipSP<\GYLowerSP
    \GYKanjiskipSP=\GYLowerSP
  \fi
  \kanjiskip=\dimexpr\GYKanjiskipSP sp\relax
}

%------------------パッケージ読み込み--------------------
% 見出し直後の段落も字下げする(\section 直後などの \noindent を無効化)
\usepackage{indentfirst}
% 本文の字下げ幅(和文の標準として 1zw を採用)
\setlength{\parindent}{1zw}



\usepackage{caption}
\usepackage{graphicx}
\usepackage{xcolor}
\usepackage{amsmath,amssymb,amsthm}				%数式align環境,数学記号,定義・定理・証明環境
\allowdisplaybreaks % align環境の途中での改ページ許可
\usepackage[subrefformat=parens,labelformat=parens]{subcaption}			%複数図をまとめる
\usepackage{tabularray}
\usepackage{booktabs}
\usepackage{url}
\usepackage[superscript]{cite}
% \usepackage{ascmac}				% 囲み
% \usepackage{color}					%色付け
% \usepackage{autonum} % 数式の番号管理
% \usepackage{relsize} % 拡大縮小
\usepackage{float} % 図の位置をその場に指定
    % \restylefloat{figure}
    % \restylefloat{table}
\usepackage{wrapfig} % 図の回り込み
% \usepackage{natbib} 
% \usepackage[sectionbib]{chapterbib}
% \usepackage{showkeys} % ラベル表示
% \usepackage{abstract}
\usepackage{fancyhdr}
\pagestyle{fancy}
% いったん全部クリア
\fancyhead{}
\fancyfoot{}
\usepackage{titlesec}
\titleformat*{\section}{\fontsize{10pt}{12pt}\bfseries}
\titleformat*{\subsection}{\fontsize{10pt}{12pt}\bfseries}
\titleformat*{\subsubsection}{\fontsize{10pt}{12pt}\bfseries}
\usepackage{lmodern}% フォント、警告対策

\titlespacing*{\section}{0pt}{3mm}{0mm}
\titlespacing*{\subsection}{0pt}{3mm}{0mm}
\titlespacing*{\subsubsection}{0pt}{3mm}{0mm}


%------------------参考文献引用を上付き(例:^{[1]})にする--------------------
% citeパッケージの[superscript]はデフォルトで括弧なしなので、括弧を付ける
\makeatletter
\def\@citess#1{\mbox{$\m@th^{\hbox{\OverciteFont{\citeleft#1\citeright}}}$}}
\makeatother

% ページ番号の書式
\newcommand{\mynumber}{--- \thepage\ ---} % 「― 1 ―」風

% 偶数ページ左上(Left-Even), 奇数ページ右上(Right-Odd)
\fancyhead[LE]{\mynumber}
\fancyhead[RO]{\mynumber}

% ヘッダー下の罫線を消したい場合
\renewcommand{\headrulewidth}{0pt}

%-------------------コマンドの定義-------------------------
\newcommand{\bm}{\boldsymbol}
\newcommand{\del}{\partial}
\newcommand{\diag}{\mathrm{diag}}
\newcommand{\theabstract}{} % まず空で用意
\newcommand{\abstracttxt}[1]{\renewcommand{\theabstract}{#1}}
\newcommand{\thesupervisorname}{} % まず空で用意
\newcommand{\thesupervisorposition}{} % まず空で用意
\newcommand{\supervisor}[2]{% 2つまとめてセット
  \renewcommand{\thesupervisorname}{#1}%
  \renewcommand{\thesupervisorposition}{#2}%
}

\theoremstyle{definition}
\newtheorem{definition}{Definition}%[section]
\newtheorem{assumption}{Assumption}%[section]
\newtheorem{theorem}{Theorem}%[section]
\newtheorem{lemma}{Lemma}%[section]
\newtheorem{corollary}{Corollary}%[section]
\newtheorem{remark}{Remark}%[section]
% \renewcommand{\proofname}{\textbf{証明}}

% subcaptionの調整
% \captionsetup[subfigure]{labelformat=simple}
% \renewcommand{\thesubfigure}{(\alph{subfigure})}


%-------------------ラベル表示の変更-----------------------------
\date{\number\year/\number\month/\number\day}
\renewcommand{\figurename}{Fig.}
\renewcommand{\tablename}{Table}
% \renewcommand{\refname}{References}
% \bibliographystyle{sicetran}
% \bibliographystyle{IEEEtran}
%\renewcommand{\theenumi}{\alph{enumi}}

% タイトルのスタイル
\makeatletter
\renewcommand{\maketitle}{
  \twocolumn[%
  \begin{center}
    \parbox{\dimexpr\textwidth-20truemm\relax}{%
      \centering
      {\fontsize{14pt}{16pt}\selectfont\bfseries
        \@title\par
      }%
    }%
  \end{center}
  \hfill\begin{minipage}{0.4\linewidth}
    \hfill\begin{tblr}{
      colspec = {l r},
      column{1} = {font=\bfseries, halign=l},
      column{2} = {halign=r},
      colsep = 0.5em,
      rows = {abovesep=0pt, belowsep=0pt, rowsep=0pt},
      row{1} = {belowsep=-2pt},
    }
      発表者 & \@author \\
      指導教員 & \thesupervisorname & \thesupervisorposition
    \end{tblr}
  \end{minipage}
  \begin{center}
    \begin{minipage}{\linewidth}
      % \hrule
      {\bfseries} \theabstract
    \end{minipage}
  \end{center}
  ]%
}
\makeatother
