%-----------------ページ余白の設定---------------------
\usepackage[top=20mm, bottom=20mm, left=15mm, right=15mm]{geometry}
%------------------パッケージ読み込み--------------------
\usepackage{caption}
\usepackage{graphicx}
\usepackage{amsmath,amssymb,amsthm}				%数式align環境,数学記号,定義・定理・証明環境
\allowdisplaybreaks % align環境の途中での改ページ許可
\usepackage[subrefformat=parens,labelformat=parens]{subcaption}			%複数図をまとめる
\usepackage{tabularray}
\usepackage{url}
% \usepackage{ascmac}				% 囲み
% \usepackage{color}					%色付け
% \usepackage{autonum} % 数式の番号管理
% \usepackage{relsize} % 拡大縮小
\usepackage{float} % 図の位置をその場に指定
    % \restylefloat{figure}
    % \restylefloat{table}
\usepackage{wrapfig} % 図の回り込み
% \usepackage{natbib} 
% \usepackage[sectionbib]{chapterbib}
% \usepackage{showkeys} % ラベル表示
% \usepackage{abstract}
\usepackage{fancyhdr}
\pagestyle{fancy}
% いったん全部クリア
\fancyhead{}
\fancyfoot{}
\usepackage{titlesec}
\titleformat*{\section}{\fontsize{10pt}{12pt}\bfseries}
\titleformat*{\subsection}{\fontsize{10pt}{12pt}\bfseries}
\titleformat*{\subsubsection}{\fontsize{10pt}{12pt}\bfseries}
\usepackage{lmodern}% フォント、警告対策

% ページ番号の書式
\newcommand{\mynumber}{--- \thepage\ ---} % 「― 1 ―」風

% 偶数ページ左上(Left-Even), 奇数ページ右上(Right-Odd)
\fancyhead[LE]{\mynumber}
\fancyhead[RO]{\mynumber}

% ヘッダー下の罫線を消したい場合
\renewcommand{\headrulewidth}{0pt}

%-------------------コマンドの定義-------------------------
\newcommand{\bm}{\boldsymbol}
\newcommand{\del}{\partial}
\newcommand{\diag}{\mathrm{diag}}
\newcommand{\theabstract}{} % まず空で用意
\newcommand{\abstracttxt}[1]{\renewcommand{\theabstract}{#1}}
\newcommand{\thesupervisorname}{} % まず空で用意
\newcommand{\thesupervisorposition}{} % まず空で用意
\newcommand{\supervisor}[2]{% 2つまとめてセット
  \renewcommand{\thesupervisorname}{#1}%
  \renewcommand{\thesupervisorposition}{#2}%
}

\theoremstyle{definition}
\newtheorem{definition}{Definition}%[section]
\newtheorem{assumption}{Assumption}%[section]
\newtheorem{theorem}{Theorem}%[section]
\newtheorem{lemma}{Lemma}%[section]
\newtheorem{corollary}{Corollary}%[section]
\newtheorem{remark}{Remark}%[section]
% \renewcommand{\proofname}{\textbf{証明}}

% subcaptionの調整
\captionsetup[subfigure]{labelformat=simple}
\renewcommand{\thesubfigure}{(\alph{subfigure})}


%-------------------ラベル表示の変更-----------------------------
\date{\number\year/\number\month/\number\day}
\renewcommand{\figurename}{Fig.\,}
\renewcommand{\tablename}{Table\,}
% \renewcommand{\refname}{References}
% \bibliographystyle{sicetran}
% \bibliographystyle{IEEEtran}
%\renewcommand{\theenumi}{\alph{enumi}}

\bibliographystyle{ssice}

% タイトルのスタイル
\makeatletter
\renewcommand{\maketitle}{
  \twocolumn[%
    \centering  
    {\fontsize{20pt}{24pt}\selectfont\bfseries
      \@title
    } \\%
  \vspace{\baselineskip}
  \begin{tblr}{
    width = { 1.0\linewidth },
    colspec = {X[4] X[1] X[1]}, % 比率指定
    rows = { rowsep=0pt },
  }
  & \bfseries 発表者 & \@author \\
  & \bfseries 指導教員 & \thesupervisorname~\thesupervisorposition
  \end{tblr}
  \begin{minipage}{\linewidth}
    \vspace{\baselineskip}
    {\bfseries Abstract: }\theabstract
    \vspace{\baselineskip}
  \end{minipage}
  ]%
}
\makeatother
