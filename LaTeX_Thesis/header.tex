%==============================================================================
% 卒業論文LaTeXテンプレート - ヘッダーファイル
% 
% このファイルには、卒論本文で使用する各種設定とコマンド定義が含まれています。
% main.tex から %==============================================================================
% 卒業論文LaTeXテンプレート - ヘッダーファイル
% 
% このファイルには、卒論本文で使用する各種設定とコマンド定義が含まれています。
% main.tex から %==============================================================================
% 卒業論文LaTeXテンプレート - ヘッダーファイル
% 
% このファイルには、卒論本文で使用する各種設定とコマンド定義が含まれています。
% main.tex から %==============================================================================
% 卒業論文LaTeXテンプレート - ヘッダーファイル
% 
% このファイルには、卒論本文で使用する各種設定とコマンド定義が含まれています。
% main.tex から \input{header} で読み込まれます。
%==============================================================================

%-----------------ページ余白の設定-----------------------------------------
% デフォルト設定(ヘッダーありのページ用)
\usepackage[top=30mm, bottom=25mm, left=20mm, right=20mm, headheight=17pt, headsep=10mm]{geometry}

% ヘッダーなしページ用の余白設定コマンド
% 使用例: \noheadergeometry → 目次ページなど → \normalgeometry で元に戻す
\newcommand{\noheadergeometry}{\newgeometry{top=40mm, bottom=25mm, left=20mm, right=20mm}}
\newcommand{\normalgeometry}{\restoregeometry}

%-----------------フォント設定-------------------------------------------------
% Times New Roman (欧文), 明朝体 (和文) - pLaTeX用
\usepackage{newtxtext}           % 欧文フォントをTimes New Romanに設定
\usepackage[amsthm]{newtxmath}   % 数式フォントもTimesに統一(定理環境対応)

% 本文を細字に設定(オプション:大学の規定に応じて調整)
% 注意: 以下をコメントアウトすると、フォント警告が消えます(標準の太さになります)
% \renewcommand{\seriesdefault}{l}        % light weight (細字)
% \AtBeginDocument{\fontseries{l}\selectfont}  % 明示的に細字を適用
%-----------------パッケージ読み込み-------------------------------------------

% 図表のキャプション設定
\usepackage{caption}
\captionsetup[figure]{
  labelfont=bf,         % ラベル(「図1」部分)を太字
  textfont=normalfont,  % 説明文は通常フォント
  labelsep=colon,       % ラベルと説明文の区切りはコロン「:」
  name=図               % 日本語で「図」と表示
}
\captionsetup[table]{
  labelfont=bf,
  textfont=normalfont,
  labelsep=colon,
  name=表               % 日本語で「表」と表示
}

% 画像・図表関連
\usepackage{graphicx}   % 画像の挿入 (\includegraphics)
\usepackage{float}      % 図表の位置を [H] で強制指定可能に
\usepackage{wrapfig}    % 文章の回り込み図

% 数式関連
\usepackage{amsmath,amsthm}  % 数式環境 (align, equation等) と定理環境
\allowdisplaybreaks          % align環境内での改ページを許可

% 複数図表の管理
\usepackage[subrefformat=parens,labelformat=parens]{subcaption}  % サブキャプション (a), (b) 形式

% 表組み
\usepackage{tabularray}  % 高機能な表作成パッケージ

% URL・ハイパーリンク
\usepackage{url}         % URLの改行処理
%-----------------ヘッダー・フッター設定---------------------------------------
\usepackage{fancyhdr}
\pagestyle{fancy}

% ヘッダー・フッター初期化
\fancyhead{}
\fancyfoot{}

% ヘッダーの設定
% 注意: \leftmark=章名、\rightmark=節名(LaTeXの仕様)
\fancyhead[L]{\rightmark}  % 左側:節名(例: 1.1 背景)
\fancyhead[R]{\leftmark}   % 右側:章名(例: 第1章 序論)

% フッター設定
\fancyfoot[C]{\thepage}    % ページ番号を中央に配置

% 罫線の設定
\renewcommand{\headrulewidth}{0.4pt}  % ヘッダー下に細い罫線
\renewcommand{\footrulewidth}{0pt}    % フッターには罫線なし

% plainスタイル(章の最初のページ)の設定
% → ヘッダーなし、フッターにページ番号のみ
\fancypagestyle{plain}{%
  \fancyhf{}%
  \fancyfoot[C]{\thepage}%
  \renewcommand{\headrulewidth}{0pt}%
  \renewcommand{\footrulewidth}{0pt}%
}

% ヘッダーのマーク定義をカスタマイズ
\renewcommand{\sectionmark}[1]{\markboth{第\arabic{section}章\quad #1}{第\arabic{section}章\quad #1}}
\renewcommand{\subsectionmark}[1]{\markright{\arabic{section}.\arabic{subsection}\quad #1}}
%-----------------見出しスタイル設定-------------------------------------------
\usepackage{titlesec}

% 章の前で改ページ(各章を新しいページから開始)
\newcommand{\sectionbreak}{\clearpage\thispagestyle{plain}}

% 章タイトル(section)のスタイル
% フォント: 明朝体(m)、サイズ: 20pt、前後の余白設定
\titleformat{\section}{\fontseries{m}\selectfont\fontsize{20pt}{18pt}\selectfont}{\thesection}{1em}{}
\titlespacing*{\section}{0pt}{20pt}{42pt}  % {左余白}{上余白}{下余白}

% 節タイトル(subsection)のスタイル
% フォント: 明朝体(m)、サイズ: 16pt
\titleformat{\subsection}{\fontseries{m}\selectfont\fontsize{16pt}{14pt}\selectfont}{\thesubsection}{1em}{}
\titlespacing*{\subsection}{0pt}{30pt}{12pt}

% 小節タイトル(subsubsection)のスタイル
% フォント: 明朝体(m)、サイズ: 14pt
\titleformat{\subsubsection}{\fontseries{m}\selectfont\fontsize{14pt}{13pt}\selectfont}{\thesubsubsection}{1em}{}
\titlespacing*{\subsubsection}{0pt}{10pt}{6pt}
%-----------------行間・段落間隔設定-------------------------------------------
\usepackage{setspace}
\setstretch{0.8}  % 行間倍率(0.8は狭め、標準は1.0、推奨は1.2程度)

\usepackage{parskip}
\setlength{\parskip}{6pt}      % 段落間の余白(6pt)
\setlength{\parindent}{1em}    % 段落の字下げを1文字分に設定

% parskipパッケージの影響を上書きして字下げを有効化
\makeatletter
\let\@afterindentfalse\@afterindenttrue
\makeatother

%-----------------カスタムコマンドの定義---------------------------------------

% 数式用ショートカットコマンド
\newcommand{\bm}{\boldsymbol}    % 数式の太字: \bm{x}
\newcommand{\del}{\partial}      % 偏微分記号: \del
\newcommand{\diag}{\mathrm{diag}} % 対角行列: \diag

% アブストラクト用の変数(報告書形式で使用)
\newcommand{\theabstract}{}

% 指導教員情報の変数
\newcommand{\thesupervisorname}{}     % 指導教員名
\newcommand{\thesupervisorposition}{} % 指導教員の役職

% 指導教員情報を設定するコマンド
% 使用例: \supervisor{梅本和希}{准教授}
\newcommand{\supervisor}[2]{%
  \renewcommand{\thesupervisorname}{#1}%
  \renewcommand{\thesupervisorposition}{#2}%
}

% 参考文献を目次に追加するコマンド
% 使用例: \makebibliography{ssice}{bibfile}
\newcommand{\makebibliography}[2]{%
  \addcontentsline{toc}{section}{\numberline{}参考文献}%
  \bibliographystyle{#1}%
  \bibliography{#2}%
}

% 謝辞セクションを目次に追加するコマンド
% 使用例: \makeacknowledgment{謝辞の本文}
\newcommand{\makeacknowledgment}[1]{%
  \section*{謝辞}%
  \addcontentsline{toc}{section}{\numberline{}謝辞}%
  #1%
}

%-----------------表紙設定用の変数---------------------------------------------
% 表紙に表示する各種情報を格納する変数
\newcommand{\theyear}{}         % 年度(例: 2025年度)
\newcommand{\theuniversity}{}   % 大学名(例: 福井大学)
\newcommand{\thedepartment}{}   % 研究科名(例: 工学研究科)
\newcommand{\thedoctype}{}      % 文書種別(例: 学士論文、修士論文)
\newcommand{\thetitlejp}{}      % 論文タイトル(日本語)
\newcommand{\thetitleen}{}      % 論文タイトル(英語)
\newcommand{\thedate}{}         % 提出日(例: 2026年2月7日)
\newcommand{\thefaculty}{}      % 学部名(例: 福井大学 工学部)
\newcommand{\thecourse}{}       % コース名(例: 機械工学コース)
\newcommand{\theauthor}{}       % 氏名
\newcommand{\thestudentid}{}    % 学籍番号

% 表紙情報を一括設定するコマンド
% 使用例: \settitlepage{2025年度}{福井大学}{工学研究科}{学士論文}{...}{...}{2026年2月7日}{山田太郎}{12345678}
\newcommand{\settitlepage}[9]{%
  \renewcommand{\theyear}{#1}%        % 年度
  \renewcommand{\theuniversity}{#2}%  % 大学名
  \renewcommand{\thedepartment}{#3}%  % 研究科名
  \renewcommand{\thedoctype}{#4}%     % 文書種別
  \renewcommand{\thetitlejp}{#5}%     % タイトル(日本語)
  \renewcommand{\thetitleen}{#6}%     % タイトル(英語)
  \renewcommand{\thedate}{#7}%        % 提出日
  \renewcommand{\theauthor}{#8}%      % 氏名
  \renewcommand{\thestudentid}{#9}%   % 学籍番号
}

%-----------------表紙生成コマンド---------------------------------------------
% 表紙を生成するコマンド
% 使用例: main.texで \maketitlepage を呼び出す
\newcommand{\maketitlepage}{%
  \begin{titlepage}
    \thispagestyle{empty}  % ページ番号を表示しない
    \begin{center}
      % 年度
      \vspace*{-10mm}
      {\fontsize{16pt}{18pt}\selectfont \theyear} \\
      \vspace{5mm}
      
      % 大学名・研究科名
      {\fontsize{16pt}{18pt}\selectfont \theuniversity\ \thedepartment} \\
      \vspace{5mm}
      
      % 文書種別(学士論文、修士論文など)
      {\fontsize{16pt}{18pt}\selectfont \thedoctype} \\
      \vspace{38mm}
      
      % 論文タイトル(日本語)
      {\fontsize{24pt}{30pt}\selectfont \bfseries \thetitlejp} \\
      \vspace{4mm}
      \noindent\rule{1.0\textwidth}{0.5pt} \\  % 区切り線
      \vspace{4mm}
      
      % 論文タイトル(英語)
      {\fontsize{20pt}{20pt}\selectfont \thetitleen} \\
      \vspace{33mm}
      
      % 提出日
      {\fontsize{16pt}{16pt}\selectfont \thedate} \\
      \vspace{10mm}
      
      % 学部・コース名
      {\fontsize{16pt}{15pt}\selectfont \thefaculty} \\
      \vspace{2mm}
      {\fontsize{16pt}{15pt}\selectfont \thecourse} \\
      \vspace{10mm}
      
      % 氏名
      {\fontsize{16pt}{15pt}\selectfont 氏名 \theauthor} \\
      \vspace{10mm}
      
      % 学籍番号
      {\fontsize{16pt}{15pt}\selectfont 学籍番号 \thestudentid} \\
      \vspace{10mm}
      
      % 指導教員
      {\fontsize{16pt}{15pt}\selectfont 指導教員 \thesupervisorname\ \thesupervisorposition} \\
    \end{center}
  \end{titlepage}
}

%-----------------定理環境の定義-----------------------------------------------
% 数学的な定義・定理・証明などの環境
% 必要に応じてコメントアウト解除: %[section] → 章ごとに番号リセット
\theoremstyle{definition}
\newtheorem{definition}{Definition}      % 定義
\newtheorem{assumption}{Assumption}      % 仮定
\newtheorem{theorem}{Theorem}            % 定理
\newtheorem{lemma}{Lemma}                % 補題
\newtheorem{corollary}{Corollary}        % 系
\newtheorem{remark}{Remark}              % 注意・備考

%-----------------サブキャプションの調整---------------------------------------
% サブ図のラベルを (a), (b) 形式に設定
\captionsetup[subfigure]{labelformat=simple}
\renewcommand{\thesubfigure}{(\alph{subfigure})}

%-----------------ラベル表示の日本語化-----------------------------------------
% 各種ラベルを日本語表記に変更
\date{\number\year/\number\month/\number\day}  % 日付形式: YYYY/MM/DD
\renewcommand{\figurename}{図}          % Figure → 図
\renewcommand{\tablename}{表}           % Table → 表
\renewcommand{\contentsname}{目次}      % Contents → 目次
\renewcommand{\refname}{参考文献}      % References → 参考文献
\renewcommand{\abstractname}{要旨}     % Abstract → 要旨

% セクション番号の表記形式を変更
\renewcommand{\thesection}{第\arabic{section}章}              % 例: 第1章
\renewcommand{\thesubsection}{\arabic{section}.\arabic{subsection}}  % 例: 1.1

%-----------------ハイパーリンク設定-------------------------------------------
% PDF内の目次・参照・URLをクリック可能なリンクに設定
% colorlinks=true: リンクに色を付ける(枠ではなく文字色)
\usepackage[dvipdfmx,colorlinks=true,linkcolor=black,citecolor=blue,urlcolor=blue]{hyperref}

%-----------------付録設定-----------------------------------------------------
\usepackage{appendix}  % 付録環境を使用可能にする

 で読み込まれます。
%==============================================================================

%-----------------ページ余白の設定-----------------------------------------
% デフォルト設定(ヘッダーありのページ用)
\usepackage[top=30mm, bottom=25mm, left=20mm, right=20mm, headheight=17pt, headsep=10mm]{geometry}

% ヘッダーなしページ用の余白設定コマンド
% 使用例: \noheadergeometry → 目次ページなど → \normalgeometry で元に戻す
\newcommand{\noheadergeometry}{\newgeometry{top=40mm, bottom=25mm, left=20mm, right=20mm}}
\newcommand{\normalgeometry}{\restoregeometry}

%-----------------フォント設定-------------------------------------------------
% Times New Roman (欧文), 明朝体 (和文) - pLaTeX用
\usepackage{newtxtext}           % 欧文フォントをTimes New Romanに設定
\usepackage[amsthm]{newtxmath}   % 数式フォントもTimesに統一(定理環境対応)

% 本文を細字に設定(オプション:大学の規定に応じて調整)
% 注意: 以下をコメントアウトすると、フォント警告が消えます(標準の太さになります)
% \renewcommand{\seriesdefault}{l}        % light weight (細字)
% \AtBeginDocument{\fontseries{l}\selectfont}  % 明示的に細字を適用
%-----------------パッケージ読み込み-------------------------------------------

% 図表のキャプション設定
\usepackage{caption}
\captionsetup[figure]{
  labelfont=bf,         % ラベル(「図1」部分)を太字
  textfont=normalfont,  % 説明文は通常フォント
  labelsep=colon,       % ラベルと説明文の区切りはコロン「:」
  name=図               % 日本語で「図」と表示
}
\captionsetup[table]{
  labelfont=bf,
  textfont=normalfont,
  labelsep=colon,
  name=表               % 日本語で「表」と表示
}

% 画像・図表関連
\usepackage{graphicx}   % 画像の挿入 (\includegraphics)
\usepackage{float}      % 図表の位置を [H] で強制指定可能に
\usepackage{wrapfig}    % 文章の回り込み図

% 数式関連
\usepackage{amsmath,amsthm}  % 数式環境 (align, equation等) と定理環境
\allowdisplaybreaks          % align環境内での改ページを許可

% 複数図表の管理
\usepackage[subrefformat=parens,labelformat=parens]{subcaption}  % サブキャプション (a), (b) 形式

% 表組み
\usepackage{tabularray}  % 高機能な表作成パッケージ

% URL・ハイパーリンク
\usepackage{url}         % URLの改行処理
%-----------------ヘッダー・フッター設定---------------------------------------
\usepackage{fancyhdr}
\pagestyle{fancy}

% ヘッダー・フッター初期化
\fancyhead{}
\fancyfoot{}

% ヘッダーの設定
% 注意: \leftmark=章名、\rightmark=節名(LaTeXの仕様)
\fancyhead[L]{\rightmark}  % 左側:節名(例: 1.1 背景)
\fancyhead[R]{\leftmark}   % 右側:章名(例: 第1章 序論)

% フッター設定
\fancyfoot[C]{\thepage}    % ページ番号を中央に配置

% 罫線の設定
\renewcommand{\headrulewidth}{0.4pt}  % ヘッダー下に細い罫線
\renewcommand{\footrulewidth}{0pt}    % フッターには罫線なし

% plainスタイル(章の最初のページ)の設定
% → ヘッダーなし、フッターにページ番号のみ
\fancypagestyle{plain}{%
  \fancyhf{}%
  \fancyfoot[C]{\thepage}%
  \renewcommand{\headrulewidth}{0pt}%
  \renewcommand{\footrulewidth}{0pt}%
}

% ヘッダーのマーク定義をカスタマイズ
\renewcommand{\sectionmark}[1]{\markboth{第\arabic{section}章\quad #1}{第\arabic{section}章\quad #1}}
\renewcommand{\subsectionmark}[1]{\markright{\arabic{section}.\arabic{subsection}\quad #1}}
%-----------------見出しスタイル設定-------------------------------------------
\usepackage{titlesec}

% 章の前で改ページ(各章を新しいページから開始)
\newcommand{\sectionbreak}{\clearpage\thispagestyle{plain}}

% 章タイトル(section)のスタイル
% フォント: 明朝体(m)、サイズ: 20pt、前後の余白設定
\titleformat{\section}{\fontseries{m}\selectfont\fontsize{20pt}{18pt}\selectfont}{\thesection}{1em}{}
\titlespacing*{\section}{0pt}{20pt}{42pt}  % {左余白}{上余白}{下余白}

% 節タイトル(subsection)のスタイル
% フォント: 明朝体(m)、サイズ: 16pt
\titleformat{\subsection}{\fontseries{m}\selectfont\fontsize{16pt}{14pt}\selectfont}{\thesubsection}{1em}{}
\titlespacing*{\subsection}{0pt}{30pt}{12pt}

% 小節タイトル(subsubsection)のスタイル
% フォント: 明朝体(m)、サイズ: 14pt
\titleformat{\subsubsection}{\fontseries{m}\selectfont\fontsize{14pt}{13pt}\selectfont}{\thesubsubsection}{1em}{}
\titlespacing*{\subsubsection}{0pt}{10pt}{6pt}
%-----------------行間・段落間隔設定-------------------------------------------
\usepackage{setspace}
\setstretch{0.8}  % 行間倍率(0.8は狭め、標準は1.0、推奨は1.2程度)

\usepackage{parskip}
\setlength{\parskip}{6pt}      % 段落間の余白(6pt)
\setlength{\parindent}{1em}    % 段落の字下げを1文字分に設定

% parskipパッケージの影響を上書きして字下げを有効化
\makeatletter
\let\@afterindentfalse\@afterindenttrue
\makeatother

%-----------------カスタムコマンドの定義---------------------------------------

% 数式用ショートカットコマンド
\newcommand{\bm}{\boldsymbol}    % 数式の太字: \bm{x}
\newcommand{\del}{\partial}      % 偏微分記号: \del
\newcommand{\diag}{\mathrm{diag}} % 対角行列: \diag

% アブストラクト用の変数(報告書形式で使用)
\newcommand{\theabstract}{}

% 指導教員情報の変数
\newcommand{\thesupervisorname}{}     % 指導教員名
\newcommand{\thesupervisorposition}{} % 指導教員の役職

% 指導教員情報を設定するコマンド
% 使用例: \supervisor{梅本和希}{准教授}
\newcommand{\supervisor}[2]{%
  \renewcommand{\thesupervisorname}{#1}%
  \renewcommand{\thesupervisorposition}{#2}%
}

% 参考文献を目次に追加するコマンド
% 使用例: \makebibliography{ssice}{bibfile}
\newcommand{\makebibliography}[2]{%
  \addcontentsline{toc}{section}{\numberline{}参考文献}%
  \bibliographystyle{#1}%
  \bibliography{#2}%
}

% 謝辞セクションを目次に追加するコマンド
% 使用例: \makeacknowledgment{謝辞の本文}
\newcommand{\makeacknowledgment}[1]{%
  \section*{謝辞}%
  \addcontentsline{toc}{section}{\numberline{}謝辞}%
  #1%
}

%-----------------表紙設定用の変数---------------------------------------------
% 表紙に表示する各種情報を格納する変数
\newcommand{\theyear}{}         % 年度(例: 2025年度)
\newcommand{\theuniversity}{}   % 大学名(例: 福井大学)
\newcommand{\thedepartment}{}   % 研究科名(例: 工学研究科)
\newcommand{\thedoctype}{}      % 文書種別(例: 学士論文、修士論文)
\newcommand{\thetitlejp}{}      % 論文タイトル(日本語)
\newcommand{\thetitleen}{}      % 論文タイトル(英語)
\newcommand{\thedate}{}         % 提出日(例: 2026年2月7日)
\newcommand{\thefaculty}{}      % 学部名(例: 福井大学 工学部)
\newcommand{\thecourse}{}       % コース名(例: 機械工学コース)
\newcommand{\theauthor}{}       % 氏名
\newcommand{\thestudentid}{}    % 学籍番号

% 表紙情報を一括設定するコマンド
% 使用例: \settitlepage{2025年度}{福井大学}{工学研究科}{学士論文}{...}{...}{2026年2月7日}{山田太郎}{12345678}
\newcommand{\settitlepage}[9]{%
  \renewcommand{\theyear}{#1}%        % 年度
  \renewcommand{\theuniversity}{#2}%  % 大学名
  \renewcommand{\thedepartment}{#3}%  % 研究科名
  \renewcommand{\thedoctype}{#4}%     % 文書種別
  \renewcommand{\thetitlejp}{#5}%     % タイトル(日本語)
  \renewcommand{\thetitleen}{#6}%     % タイトル(英語)
  \renewcommand{\thedate}{#7}%        % 提出日
  \renewcommand{\theauthor}{#8}%      % 氏名
  \renewcommand{\thestudentid}{#9}%   % 学籍番号
}

%-----------------表紙生成コマンド---------------------------------------------
% 表紙を生成するコマンド
% 使用例: main.texで \maketitlepage を呼び出す
\newcommand{\maketitlepage}{%
  \begin{titlepage}
    \thispagestyle{empty}  % ページ番号を表示しない
    \begin{center}
      % 年度
      \vspace*{-10mm}
      {\fontsize{16pt}{18pt}\selectfont \theyear} \\
      \vspace{5mm}
      
      % 大学名・研究科名
      {\fontsize{16pt}{18pt}\selectfont \theuniversity\ \thedepartment} \\
      \vspace{5mm}
      
      % 文書種別(学士論文、修士論文など)
      {\fontsize{16pt}{18pt}\selectfont \thedoctype} \\
      \vspace{38mm}
      
      % 論文タイトル(日本語)
      {\fontsize{24pt}{30pt}\selectfont \bfseries \thetitlejp} \\
      \vspace{4mm}
      \noindent\rule{1.0\textwidth}{0.5pt} \\  % 区切り線
      \vspace{4mm}
      
      % 論文タイトル(英語)
      {\fontsize{20pt}{20pt}\selectfont \thetitleen} \\
      \vspace{33mm}
      
      % 提出日
      {\fontsize{16pt}{16pt}\selectfont \thedate} \\
      \vspace{10mm}
      
      % 学部・コース名
      {\fontsize{16pt}{15pt}\selectfont \thefaculty} \\
      \vspace{2mm}
      {\fontsize{16pt}{15pt}\selectfont \thecourse} \\
      \vspace{10mm}
      
      % 氏名
      {\fontsize{16pt}{15pt}\selectfont 氏名 \theauthor} \\
      \vspace{10mm}
      
      % 学籍番号
      {\fontsize{16pt}{15pt}\selectfont 学籍番号 \thestudentid} \\
      \vspace{10mm}
      
      % 指導教員
      {\fontsize{16pt}{15pt}\selectfont 指導教員 \thesupervisorname\ \thesupervisorposition} \\
    \end{center}
  \end{titlepage}
}

%-----------------定理環境の定義-----------------------------------------------
% 数学的な定義・定理・証明などの環境
% 必要に応じてコメントアウト解除: %[section] → 章ごとに番号リセット
\theoremstyle{definition}
\newtheorem{definition}{Definition}      % 定義
\newtheorem{assumption}{Assumption}      % 仮定
\newtheorem{theorem}{Theorem}            % 定理
\newtheorem{lemma}{Lemma}                % 補題
\newtheorem{corollary}{Corollary}        % 系
\newtheorem{remark}{Remark}              % 注意・備考

%-----------------サブキャプションの調整---------------------------------------
% サブ図のラベルを (a), (b) 形式に設定
\captionsetup[subfigure]{labelformat=simple}
\renewcommand{\thesubfigure}{(\alph{subfigure})}

%-----------------ラベル表示の日本語化-----------------------------------------
% 各種ラベルを日本語表記に変更
\date{\number\year/\number\month/\number\day}  % 日付形式: YYYY/MM/DD
\renewcommand{\figurename}{図}          % Figure → 図
\renewcommand{\tablename}{表}           % Table → 表
\renewcommand{\contentsname}{目次}      % Contents → 目次
\renewcommand{\refname}{参考文献}      % References → 参考文献
\renewcommand{\abstractname}{要旨}     % Abstract → 要旨

% セクション番号の表記形式を変更
\renewcommand{\thesection}{第\arabic{section}章}              % 例: 第1章
\renewcommand{\thesubsection}{\arabic{section}.\arabic{subsection}}  % 例: 1.1

%-----------------ハイパーリンク設定-------------------------------------------
% PDF内の目次・参照・URLをクリック可能なリンクに設定
% colorlinks=true: リンクに色を付ける(枠ではなく文字色)
\usepackage[dvipdfmx,colorlinks=true,linkcolor=black,citecolor=blue,urlcolor=blue]{hyperref}

%-----------------付録設定-----------------------------------------------------
\usepackage{appendix}  % 付録環境を使用可能にする

 で読み込まれます。
%==============================================================================

%-----------------ページ余白の設定-----------------------------------------
% デフォルト設定(ヘッダーありのページ用)
\usepackage[top=30mm, bottom=25mm, left=20mm, right=20mm, headheight=17pt, headsep=10mm]{geometry}

% ヘッダーなしページ用の余白設定コマンド
% 使用例: \noheadergeometry → 目次ページなど → \normalgeometry で元に戻す
\newcommand{\noheadergeometry}{\newgeometry{top=40mm, bottom=25mm, left=20mm, right=20mm}}
\newcommand{\normalgeometry}{\restoregeometry}

%-----------------フォント設定-------------------------------------------------
% Times New Roman (欧文), 明朝体 (和文) - pLaTeX用
\usepackage{newtxtext}           % 欧文フォントをTimes New Romanに設定
\usepackage[amsthm]{newtxmath}   % 数式フォントもTimesに統一(定理環境対応)

% 本文を細字に設定(オプション:大学の規定に応じて調整)
% 注意: 以下をコメントアウトすると、フォント警告が消えます(標準の太さになります)
% \renewcommand{\seriesdefault}{l}        % light weight (細字)
% \AtBeginDocument{\fontseries{l}\selectfont}  % 明示的に細字を適用
%-----------------パッケージ読み込み-------------------------------------------

% 図表のキャプション設定
\usepackage{caption}
\captionsetup[figure]{
  labelfont=bf,         % ラベル(「図1」部分)を太字
  textfont=normalfont,  % 説明文は通常フォント
  labelsep=colon,       % ラベルと説明文の区切りはコロン「:」
  name=図               % 日本語で「図」と表示
}
\captionsetup[table]{
  labelfont=bf,
  textfont=normalfont,
  labelsep=colon,
  name=表               % 日本語で「表」と表示
}

% 画像・図表関連
\usepackage{graphicx}   % 画像の挿入 (\includegraphics)
\usepackage{float}      % 図表の位置を [H] で強制指定可能に
\usepackage{wrapfig}    % 文章の回り込み図

% 数式関連
\usepackage{amsmath,amsthm}  % 数式環境 (align, equation等) と定理環境
\allowdisplaybreaks          % align環境内での改ページを許可

% 複数図表の管理
\usepackage[subrefformat=parens,labelformat=parens]{subcaption}  % サブキャプション (a), (b) 形式

% 表組み
\usepackage{tabularray}  % 高機能な表作成パッケージ

% URL・ハイパーリンク
\usepackage{url}         % URLの改行処理
%-----------------ヘッダー・フッター設定---------------------------------------
\usepackage{fancyhdr}
\pagestyle{fancy}

% ヘッダー・フッター初期化
\fancyhead{}
\fancyfoot{}

% ヘッダーの設定
% 注意: \leftmark=章名、\rightmark=節名(LaTeXの仕様)
\fancyhead[L]{\rightmark}  % 左側:節名(例: 1.1 背景)
\fancyhead[R]{\leftmark}   % 右側:章名(例: 第1章 序論)

% フッター設定
\fancyfoot[C]{\thepage}    % ページ番号を中央に配置

% 罫線の設定
\renewcommand{\headrulewidth}{0.4pt}  % ヘッダー下に細い罫線
\renewcommand{\footrulewidth}{0pt}    % フッターには罫線なし

% plainスタイル(章の最初のページ)の設定
% → ヘッダーなし、フッターにページ番号のみ
\fancypagestyle{plain}{%
  \fancyhf{}%
  \fancyfoot[C]{\thepage}%
  \renewcommand{\headrulewidth}{0pt}%
  \renewcommand{\footrulewidth}{0pt}%
}

% ヘッダーのマーク定義をカスタマイズ
\renewcommand{\sectionmark}[1]{\markboth{第\arabic{section}章\quad #1}{第\arabic{section}章\quad #1}}
\renewcommand{\subsectionmark}[1]{\markright{\arabic{section}.\arabic{subsection}\quad #1}}
%-----------------見出しスタイル設定-------------------------------------------
\usepackage{titlesec}

% 章の前で改ページ(各章を新しいページから開始)
\newcommand{\sectionbreak}{\clearpage\thispagestyle{plain}}

% 章タイトル(section)のスタイル
% フォント: 明朝体(m)、サイズ: 20pt、前後の余白設定
\titleformat{\section}{\fontseries{m}\selectfont\fontsize{20pt}{18pt}\selectfont}{\thesection}{1em}{}
\titlespacing*{\section}{0pt}{20pt}{42pt}  % {左余白}{上余白}{下余白}

% 節タイトル(subsection)のスタイル
% フォント: 明朝体(m)、サイズ: 16pt
\titleformat{\subsection}{\fontseries{m}\selectfont\fontsize{16pt}{14pt}\selectfont}{\thesubsection}{1em}{}
\titlespacing*{\subsection}{0pt}{30pt}{12pt}

% 小節タイトル(subsubsection)のスタイル
% フォント: 明朝体(m)、サイズ: 14pt
\titleformat{\subsubsection}{\fontseries{m}\selectfont\fontsize{14pt}{13pt}\selectfont}{\thesubsubsection}{1em}{}
\titlespacing*{\subsubsection}{0pt}{10pt}{6pt}
%-----------------行間・段落間隔設定-------------------------------------------
\usepackage{setspace}
\setstretch{0.8}  % 行間倍率(0.8は狭め、標準は1.0、推奨は1.2程度)

\usepackage{parskip}
\setlength{\parskip}{6pt}      % 段落間の余白(6pt)
\setlength{\parindent}{1em}    % 段落の字下げを1文字分に設定

% parskipパッケージの影響を上書きして字下げを有効化
\makeatletter
\let\@afterindentfalse\@afterindenttrue
\makeatother

%-----------------カスタムコマンドの定義---------------------------------------

% 数式用ショートカットコマンド
\newcommand{\bm}{\boldsymbol}    % 数式の太字: \bm{x}
\newcommand{\del}{\partial}      % 偏微分記号: \del
\newcommand{\diag}{\mathrm{diag}} % 対角行列: \diag

% アブストラクト用の変数(報告書形式で使用)
\newcommand{\theabstract}{}

% 指導教員情報の変数
\newcommand{\thesupervisorname}{}     % 指導教員名
\newcommand{\thesupervisorposition}{} % 指導教員の役職

% 指導教員情報を設定するコマンド
% 使用例: \supervisor{梅本和希}{准教授}
\newcommand{\supervisor}[2]{%
  \renewcommand{\thesupervisorname}{#1}%
  \renewcommand{\thesupervisorposition}{#2}%
}

% 参考文献を目次に追加するコマンド
% 使用例: \makebibliography{ssice}{bibfile}
\newcommand{\makebibliography}[2]{%
  \addcontentsline{toc}{section}{\numberline{}参考文献}%
  \bibliographystyle{#1}%
  \bibliography{#2}%
}

% 謝辞セクションを目次に追加するコマンド
% 使用例: \makeacknowledgment{謝辞の本文}
\newcommand{\makeacknowledgment}[1]{%
  \section*{謝辞}%
  \addcontentsline{toc}{section}{\numberline{}謝辞}%
  #1%
}

%-----------------表紙設定用の変数---------------------------------------------
% 表紙に表示する各種情報を格納する変数
\newcommand{\theyear}{}         % 年度(例: 2025年度)
\newcommand{\theuniversity}{}   % 大学名(例: 福井大学)
\newcommand{\thedepartment}{}   % 研究科名(例: 工学研究科)
\newcommand{\thedoctype}{}      % 文書種別(例: 学士論文、修士論文)
\newcommand{\thetitlejp}{}      % 論文タイトル(日本語)
\newcommand{\thetitleen}{}      % 論文タイトル(英語)
\newcommand{\thedate}{}         % 提出日(例: 2026年2月7日)
\newcommand{\thefaculty}{}      % 学部名(例: 福井大学 工学部)
\newcommand{\thecourse}{}       % コース名(例: 機械工学コース)
\newcommand{\theauthor}{}       % 氏名
\newcommand{\thestudentid}{}    % 学籍番号

% 表紙情報を一括設定するコマンド
% 使用例: \settitlepage{2025年度}{福井大学}{工学研究科}{学士論文}{...}{...}{2026年2月7日}{山田太郎}{12345678}
\newcommand{\settitlepage}[9]{%
  \renewcommand{\theyear}{#1}%        % 年度
  \renewcommand{\theuniversity}{#2}%  % 大学名
  \renewcommand{\thedepartment}{#3}%  % 研究科名
  \renewcommand{\thedoctype}{#4}%     % 文書種別
  \renewcommand{\thetitlejp}{#5}%     % タイトル(日本語)
  \renewcommand{\thetitleen}{#6}%     % タイトル(英語)
  \renewcommand{\thedate}{#7}%        % 提出日
  \renewcommand{\theauthor}{#8}%      % 氏名
  \renewcommand{\thestudentid}{#9}%   % 学籍番号
}

%-----------------表紙生成コマンド---------------------------------------------
% 表紙を生成するコマンド
% 使用例: main.texで \maketitlepage を呼び出す
\newcommand{\maketitlepage}{%
  \begin{titlepage}
    \thispagestyle{empty}  % ページ番号を表示しない
    \begin{center}
      % 年度
      \vspace*{-10mm}
      {\fontsize{16pt}{18pt}\selectfont \theyear} \\
      \vspace{5mm}
      
      % 大学名・研究科名
      {\fontsize{16pt}{18pt}\selectfont \theuniversity\ \thedepartment} \\
      \vspace{5mm}
      
      % 文書種別(学士論文、修士論文など)
      {\fontsize{16pt}{18pt}\selectfont \thedoctype} \\
      \vspace{38mm}
      
      % 論文タイトル(日本語)
      {\fontsize{24pt}{30pt}\selectfont \bfseries \thetitlejp} \\
      \vspace{4mm}
      \noindent\rule{1.0\textwidth}{0.5pt} \\  % 区切り線
      \vspace{4mm}
      
      % 論文タイトル(英語)
      {\fontsize{20pt}{20pt}\selectfont \thetitleen} \\
      \vspace{33mm}
      
      % 提出日
      {\fontsize{16pt}{16pt}\selectfont \thedate} \\
      \vspace{10mm}
      
      % 学部・コース名
      {\fontsize{16pt}{15pt}\selectfont \thefaculty} \\
      \vspace{2mm}
      {\fontsize{16pt}{15pt}\selectfont \thecourse} \\
      \vspace{10mm}
      
      % 氏名
      {\fontsize{16pt}{15pt}\selectfont 氏名 \theauthor} \\
      \vspace{10mm}
      
      % 学籍番号
      {\fontsize{16pt}{15pt}\selectfont 学籍番号 \thestudentid} \\
      \vspace{10mm}
      
      % 指導教員
      {\fontsize{16pt}{15pt}\selectfont 指導教員 \thesupervisorname\ \thesupervisorposition} \\
    \end{center}
  \end{titlepage}
}

%-----------------定理環境の定義-----------------------------------------------
% 数学的な定義・定理・証明などの環境
% 必要に応じてコメントアウト解除: %[section] → 章ごとに番号リセット
\theoremstyle{definition}
\newtheorem{definition}{Definition}      % 定義
\newtheorem{assumption}{Assumption}      % 仮定
\newtheorem{theorem}{Theorem}            % 定理
\newtheorem{lemma}{Lemma}                % 補題
\newtheorem{corollary}{Corollary}        % 系
\newtheorem{remark}{Remark}              % 注意・備考

%-----------------サブキャプションの調整---------------------------------------
% サブ図のラベルを (a), (b) 形式に設定
\captionsetup[subfigure]{labelformat=simple}
\renewcommand{\thesubfigure}{(\alph{subfigure})}

%-----------------ラベル表示の日本語化-----------------------------------------
% 各種ラベルを日本語表記に変更
\date{\number\year/\number\month/\number\day}  % 日付形式: YYYY/MM/DD
\renewcommand{\figurename}{図}          % Figure → 図
\renewcommand{\tablename}{表}           % Table → 表
\renewcommand{\contentsname}{目次}      % Contents → 目次
\renewcommand{\refname}{参考文献}      % References → 参考文献
\renewcommand{\abstractname}{要旨}     % Abstract → 要旨

% セクション番号の表記形式を変更
\renewcommand{\thesection}{第\arabic{section}章}              % 例: 第1章
\renewcommand{\thesubsection}{\arabic{section}.\arabic{subsection}}  % 例: 1.1

%-----------------ハイパーリンク設定-------------------------------------------
% PDF内の目次・参照・URLをクリック可能なリンクに設定
% colorlinks=true: リンクに色を付ける(枠ではなく文字色)
\usepackage[dvipdfmx,colorlinks=true,linkcolor=black,citecolor=blue,urlcolor=blue]{hyperref}

%-----------------付録設定-----------------------------------------------------
\usepackage{appendix}  % 付録環境を使用可能にする

 で読み込まれます。
%==============================================================================

%-----------------ページ余白の設定-----------------------------------------
% デフォルト設定(ヘッダーありのページ用)
\usepackage[top=30mm, bottom=25mm, left=20mm, right=20mm, headheight=17pt, headsep=10mm]{geometry}

% ヘッダーなしページ用の余白設定コマンド
% 使用例: \noheadergeometry → 目次ページなど → \normalgeometry で元に戻す
\newcommand{\noheadergeometry}{\newgeometry{top=40mm, bottom=25mm, left=20mm, right=20mm}}
\newcommand{\normalgeometry}{\restoregeometry}

%-----------------フォント設定-------------------------------------------------
% Times New Roman (欧文), 明朝体 (和文) - pLaTeX用
\usepackage{newtxtext}           % 欧文フォントをTimes New Romanに設定
\usepackage[amsthm]{newtxmath}   % 数式フォントもTimesに統一(定理環境対応)

% 本文を細字に設定(オプション:大学の規定に応じて調整)
% 注意: 以下をコメントアウトすると、フォント警告が消えます(標準の太さになります)
% \renewcommand{\seriesdefault}{l}        % light weight (細字)
% \AtBeginDocument{\fontseries{l}\selectfont}  % 明示的に細字を適用
%-----------------パッケージ読み込み-------------------------------------------

% 図表のキャプション設定
\usepackage{caption}
\captionsetup[figure]{
  labelfont=bf,         % ラベル(「図1」部分)を太字
  textfont=normalfont,  % 説明文は通常フォント
  labelsep=colon,       % ラベルと説明文の区切りはコロン「:」
  name=図               % 日本語で「図」と表示
}
\captionsetup[table]{
  labelfont=bf,
  textfont=normalfont,
  labelsep=colon,
  name=表               % 日本語で「表」と表示
}

% 画像・図表関連
\usepackage{graphicx}   % 画像の挿入 (\includegraphics)
\usepackage{float}      % 図表の位置を [H] で強制指定可能に
\usepackage{wrapfig}    % 文章の回り込み図

% 数式関連
\usepackage{amsmath,amsthm}  % 数式環境 (align, equation等) と定理環境
\allowdisplaybreaks          % align環境内での改ページを許可

% 複数図表の管理
\usepackage[subrefformat=parens,labelformat=parens]{subcaption}  % サブキャプション (a), (b) 形式

% 表組み
\usepackage{tabularray}  % 高機能な表作成パッケージ

% URL・ハイパーリンク
\usepackage{url}         % URLの改行処理
%-----------------ヘッダー・フッター設定---------------------------------------
\usepackage{fancyhdr}
\pagestyle{fancy}

% ヘッダー・フッター初期化
\fancyhead{}
\fancyfoot{}

% ヘッダーの設定
% 注意: \leftmark=章名、\rightmark=節名(LaTeXの仕様)
\fancyhead[L]{\rightmark}  % 左側:節名(例: 1.1 背景)
\fancyhead[R]{\leftmark}   % 右側:章名(例: 第1章 序論)

% フッター設定
\fancyfoot[C]{\thepage}    % ページ番号を中央に配置

% 罫線の設定
\renewcommand{\headrulewidth}{0.4pt}  % ヘッダー下に細い罫線
\renewcommand{\footrulewidth}{0pt}    % フッターには罫線なし

% plainスタイル(章の最初のページ)の設定
% → ヘッダーなし、フッターにページ番号のみ
\fancypagestyle{plain}{%
  \fancyhf{}%
  \fancyfoot[C]{\thepage}%
  \renewcommand{\headrulewidth}{0pt}%
  \renewcommand{\footrulewidth}{0pt}%
}

% ヘッダーのマーク定義をカスタマイズ
\renewcommand{\sectionmark}[1]{\markboth{第\arabic{section}章\quad #1}{第\arabic{section}章\quad #1}}
\renewcommand{\subsectionmark}[1]{\markright{\arabic{section}.\arabic{subsection}\quad #1}}
%-----------------見出しスタイル設定-------------------------------------------
\usepackage{titlesec}

% 章の前で改ページ(各章を新しいページから開始)
\newcommand{\sectionbreak}{\clearpage\thispagestyle{plain}}

% 章タイトル(section)のスタイル
% フォント: 明朝体(m)、サイズ: 20pt、前後の余白設定
\titleformat{\section}{\fontseries{m}\selectfont\fontsize{20pt}{18pt}\selectfont}{\thesection}{1em}{}
\titlespacing*{\section}{0pt}{20pt}{42pt}  % {左余白}{上余白}{下余白}

% 節タイトル(subsection)のスタイル
% フォント: 明朝体(m)、サイズ: 16pt
\titleformat{\subsection}{\fontseries{m}\selectfont\fontsize{16pt}{14pt}\selectfont}{\thesubsection}{1em}{}
\titlespacing*{\subsection}{0pt}{30pt}{12pt}

% 小節タイトル(subsubsection)のスタイル
% フォント: 明朝体(m)、サイズ: 14pt
\titleformat{\subsubsection}{\fontseries{m}\selectfont\fontsize{14pt}{13pt}\selectfont}{\thesubsubsection}{1em}{}
\titlespacing*{\subsubsection}{0pt}{10pt}{6pt}
%-----------------行間・段落間隔設定-------------------------------------------
\usepackage{setspace}
\setstretch{0.8}  % 行間倍率(0.8は狭め、標準は1.0、推奨は1.2程度)

\usepackage{parskip}
\setlength{\parskip}{6pt}      % 段落間の余白(6pt)
\setlength{\parindent}{1em}    % 段落の字下げを1文字分に設定

% parskipパッケージの影響を上書きして字下げを有効化
\makeatletter
\let\@afterindentfalse\@afterindenttrue
\makeatother

%-----------------カスタムコマンドの定義---------------------------------------

% 数式用ショートカットコマンド
\newcommand{\bm}{\boldsymbol}    % 数式の太字: \bm{x}
\newcommand{\del}{\partial}      % 偏微分記号: \del
\newcommand{\diag}{\mathrm{diag}} % 対角行列: \diag

% アブストラクト用の変数(報告書形式で使用)
\newcommand{\theabstract}{}

% 指導教員情報の変数
\newcommand{\thesupervisorname}{}     % 指導教員名
\newcommand{\thesupervisorposition}{} % 指導教員の役職

% 指導教員情報を設定するコマンド
% 使用例: \supervisor{梅本和希}{准教授}
\newcommand{\supervisor}[2]{%
  \renewcommand{\thesupervisorname}{#1}%
  \renewcommand{\thesupervisorposition}{#2}%
}

% 参考文献を目次に追加するコマンド
% 使用例: \makebibliography{ssice}{bibfile}
\newcommand{\makebibliography}[2]{%
  \addcontentsline{toc}{section}{\numberline{}参考文献}%
  \bibliographystyle{#1}%
  \bibliography{#2}%
}

% 謝辞セクションを目次に追加するコマンド
% 使用例: \makeacknowledgment{謝辞の本文}
\newcommand{\makeacknowledgment}[1]{%
  \section*{謝辞}%
  \addcontentsline{toc}{section}{\numberline{}謝辞}%
  #1%
}

%-----------------表紙設定用の変数---------------------------------------------
% 表紙に表示する各種情報を格納する変数
\newcommand{\theyear}{}         % 年度(例: 2025年度)
\newcommand{\theuniversity}{}   % 大学名(例: 福井大学)
\newcommand{\thedepartment}{}   % 研究科名(例: 工学研究科)
\newcommand{\thedoctype}{}      % 文書種別(例: 学士論文、修士論文)
\newcommand{\thetitlejp}{}      % 論文タイトル(日本語)
\newcommand{\thetitleen}{}      % 論文タイトル(英語)
\newcommand{\thedate}{}         % 提出日(例: 2026年2月7日)
\newcommand{\thefaculty}{}      % 学部名(例: 福井大学 工学部)
\newcommand{\thecourse}{}       % コース名(例: 機械工学コース)
\newcommand{\theauthor}{}       % 氏名
\newcommand{\thestudentid}{}    % 学籍番号

% 表紙情報を一括設定するコマンド
% 使用例: \settitlepage{2025年度}{福井大学}{工学研究科}{学士論文}{...}{...}{2026年2月7日}{山田太郎}{12345678}
\newcommand{\settitlepage}[9]{%
  \renewcommand{\theyear}{#1}%        % 年度
  \renewcommand{\theuniversity}{#2}%  % 大学名
  \renewcommand{\thedepartment}{#3}%  % 研究科名
  \renewcommand{\thedoctype}{#4}%     % 文書種別
  \renewcommand{\thetitlejp}{#5}%     % タイトル(日本語)
  \renewcommand{\thetitleen}{#6}%     % タイトル(英語)
  \renewcommand{\thedate}{#7}%        % 提出日
  \renewcommand{\theauthor}{#8}%      % 氏名
  \renewcommand{\thestudentid}{#9}%   % 学籍番号
}

%-----------------表紙生成コマンド---------------------------------------------
% 表紙を生成するコマンド
% 使用例: main.texで \maketitlepage を呼び出す
\newcommand{\maketitlepage}{%
  \begin{titlepage}
    \thispagestyle{empty}  % ページ番号を表示しない
    \begin{center}
      % 年度
      \vspace*{-10mm}
      {\fontsize{16pt}{18pt}\selectfont \theyear} \\
      \vspace{5mm}
      
      % 大学名・研究科名
      {\fontsize{16pt}{18pt}\selectfont \theuniversity\ \thedepartment} \\
      \vspace{5mm}
      
      % 文書種別(学士論文、修士論文など)
      {\fontsize{16pt}{18pt}\selectfont \thedoctype} \\
      \vspace{38mm}
      
      % 論文タイトル(日本語)
      {\fontsize{24pt}{30pt}\selectfont \bfseries \thetitlejp} \\
      \vspace{4mm}
      \noindent\rule{1.0\textwidth}{0.5pt} \\  % 区切り線
      \vspace{4mm}
      
      % 論文タイトル(英語)
      {\fontsize{20pt}{20pt}\selectfont \thetitleen} \\
      \vspace{33mm}
      
      % 提出日
      {\fontsize{16pt}{16pt}\selectfont \thedate} \\
      \vspace{10mm}
      
      % 学部・コース名
      {\fontsize{16pt}{15pt}\selectfont \thefaculty} \\
      \vspace{2mm}
      {\fontsize{16pt}{15pt}\selectfont \thecourse} \\
      \vspace{10mm}
      
      % 氏名
      {\fontsize{16pt}{15pt}\selectfont 氏名 \theauthor} \\
      \vspace{10mm}
      
      % 学籍番号
      {\fontsize{16pt}{15pt}\selectfont 学籍番号 \thestudentid} \\
      \vspace{10mm}
      
      % 指導教員
      {\fontsize{16pt}{15pt}\selectfont 指導教員 \thesupervisorname\ \thesupervisorposition} \\
    \end{center}
  \end{titlepage}
}

%-----------------定理環境の定義-----------------------------------------------
% 数学的な定義・定理・証明などの環境
% 必要に応じてコメントアウト解除: %[section] → 章ごとに番号リセット
\theoremstyle{definition}
\newtheorem{definition}{Definition}      % 定義
\newtheorem{assumption}{Assumption}      % 仮定
\newtheorem{theorem}{Theorem}            % 定理
\newtheorem{lemma}{Lemma}                % 補題
\newtheorem{corollary}{Corollary}        % 系
\newtheorem{remark}{Remark}              % 注意・備考

%-----------------サブキャプションの調整---------------------------------------
% サブ図のラベルを (a), (b) 形式に設定
\captionsetup[subfigure]{labelformat=simple}
\renewcommand{\thesubfigure}{(\alph{subfigure})}

%-----------------ラベル表示の日本語化-----------------------------------------
% 各種ラベルを日本語表記に変更
\date{\number\year/\number\month/\number\day}  % 日付形式: YYYY/MM/DD
\renewcommand{\figurename}{図}          % Figure → 図
\renewcommand{\tablename}{表}           % Table → 表
\renewcommand{\contentsname}{目次}      % Contents → 目次
\renewcommand{\refname}{参考文献}      % References → 参考文献
\renewcommand{\abstractname}{要旨}     % Abstract → 要旨

% セクション番号の表記形式を変更
\renewcommand{\thesection}{第\arabic{section}章}              % 例: 第1章
\renewcommand{\thesubsection}{\arabic{section}.\arabic{subsection}}  % 例: 1.1

%-----------------ハイパーリンク設定-------------------------------------------
% PDF内の目次・参照・URLをクリック可能なリンクに設定
% colorlinks=true: リンクに色を付ける(枠ではなく文字色)
\usepackage[dvipdfmx,colorlinks=true,linkcolor=black,citecolor=blue,urlcolor=blue]{hyperref}

%-----------------付録設定-----------------------------------------------------
\usepackage{appendix}  % 付録環境を使用可能にする

